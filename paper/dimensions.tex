\documentclass[letterpaper, modern]{aastex62}

% text
\newcommand{\documentname}{\textsl{Article}}
\newcommand{\sectionname}{Section}

% math
\newcommand{\given}{\,|\,}
\newcommand{\trans}[1]{{#1}^{\!\mathsf{T}}}

\begin{document}\sloppy\sloppypar\raggedbottom\frenchspacing

\title{What is the dimensionality of element-abundance space?}

\begin{abstract}
foo and bar.
\end{abstract}

\section{Introduction}

Types of supernovae are few in number.

Types of nucleosynthetic process are few in number.

Open clusters show great homogeneity, so intra-cloud mixing must be good.

Therefore, stars cannot span a huge space in chemical abundances.

Indeed, this conclusion has been arrived at previously in theory, and
the dimensionality has been measured before in data. The measurements to
date suffer from significant issues: precision, methodology.

Here we test this with the best possible data set at the present day:
All the reasons these twins are good.

\section{Generalities}

This \documentname\ wants to ask the question:
\emph{What is the observed dimensionality of chemical-abundance
  space?}
Unfortunately---and like all important questions---this question is
ill posed.
Our problem in this \sectionname\ is to write down well-posed questions
that go some way towards answering the ill-posed question.

\section{Data}

Bedell!

Show (some of) the 900 plots here; they justify the precision claims.
I would say all 32 against feh and then all 32 against something that
clearly demonstrates the precision.

Ness cluster sample.

All of APOGEE RC sample?

\section{Experiments and results}

PCA first; it is highly suggestive. Take then the opportunity to cricize PCA.

Is there any sense in which the eigen-directions look like SNe yields?
Or nucleosynthetic yields from processes?

\section{Discussion}

\appendix
\section{Linear vs log space}

Almost all work on element abundances works in log (base 10), and most is
relative to Solar.
If we assume that there are a finite number $K$ of nucleosynthetic pathways,
and that each of those pathways produces elements in nearly fixed
ratios of atoms or mass, then the abundances will lie in a $K$-dimensional
subspace (or lower).
This subspace will only appear as a strictly linear subspace (that is, strictly
spanned by $K$ vectors) in linear ratios of elements.
These linear ratios can be mass ratios or atomic number ratios, and they can be ratioed
to other elements or Solar abundances, but without taking logarithms.

If the dimensionality of the element abundance linear subspace is $K$, and the
subspace is searched for with a linear algorithm but in the logarithmic element
abundances, then in principle more than $K$ significant dimensions or vectors
will be required to span the space of the data.
This is because a linear subspace in the linear data will obtain some curvature
when the logs are taken, and that curvature will be in directions that are orthogonal
to the subspace (in general).

For these reasons, in this paper, when we perform the dimensionality reduction,
we take the (log) abundances back to linear, perform the dimensionality reduction
in the linear space, and then transform the output so that it can be interpreted
logarithmically.
This latter transformation---of eigenvectors in the linear space to equivalents
in the log space---works as follows:

[Give equation and derivaton here.]

The amusing thing about all this, in the context of the project reported in this paper,
is that the spread in chemical abundances is very small in the sample.
In the limit of infinitesimal element-abundance spreads, there will be no differences
between the properly transformed linear eigenvectors and logarithmic space eigenvectors.
We demonstrate this in Figure~XXX.

\end{document}
