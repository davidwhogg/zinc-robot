\documentclass[letterpaper, modern]{aastex62}

% text
\newcommand{\documentname}{\textsl{Article}}
\newcommand{\sectionname}{Section}
\newcommand{\project}[1]{\textsl{#1}}
\newcommand{\acronym}[1]{{\small{#1}}}
\newcommand{\APOGEE}{\project{\acronym{APOGEE}}}
\newcommand{\HARPS}{\project{\acronym{HARPS}}}
\newcommand{\MIKE}{\project{\acronym{MIKE}}}

% math
\newcommand{\given}{\,|\,}
\newcommand{\trans}[1]{{#1}^{\!\mathsf{T}}}

\begin{document}\sloppy\sloppypar\raggedbottom\frenchspacing

\title{What is the dimensionality of element-abundance space?}
\author{MB}
\author{DWH}

\begin{abstract}
%Context
The concept of chemical tagging is predicated on a (near future) capability
of identifying (at least probabilistically) co-eval stars by particular
element-abundance patterns.
At the same time, the finite number of nucleosynthetic pathways (and secondary
processes for modifying abundances) suggests that the diversity in
element-abundance space might be limited, especially in dimensionality.
It is critical to understand this diversity, and also how precisely
we can discriminate similar and dissimilar pairs of stars in realistic data.
%Aims
Here we look at the dimensionality of abundance space in three different
samples: A small set of extremely precisely (and differentially) measured
Solar Twins measured by \HARPS\ and \MIKE,
an even smaller set of star clusters measured by \APOGEE\ (and something?),
and a large set of red-clump (RC) stars measured by \APOGEE.
Any dimensionality reduction also, in its tangent space, delivers an empirical
test of any claimed measurement precision.
%Methods
We look at dimensionality and discrimination by two different methods.
In one, we perform principal components analysis and look at the number of
components that show excess scatter over noise, for different assumptions about
measurement noise.
In the other, we look at how one element abundance can be predicted from combinations
of other element abundances.
%Results
We find that there are only 4 to 15 dimensions or eigen-directions with
non-zero variance in the abundance space of the Solar Twins,
which have 30 elements precisely measured, but which are only sensitive to disk
stars that have radially migrated to the Solar neighborhood.
We find XXX dimensions for the \APOGEE\ RC abundances, which are measured
all over the Galaxy but in a smaller set of elements.
We confirm measuremeent uncertainties in the YYY-dex range for the Solar Twins,
ZZZ-dex for the star clusters, and WWW-dex for the RC stars.
\end{abstract}

\section{Introduction}

There is a great hope for Milky Way astrophysics that it will be possible
to identify stars that were born together by their particular element-abundance
peculiarities.
These ``chemical tags'' would permit us to unravel dynamical evolution in the
Milky Way, including in principle radial migration, mass assembly, and gas accretion and
star formation.
It is a beautiful idea, because there are many elements measured now for
many stars, and even with coarse information per element, the discriminatory
power of a measurement of chemical abundances is na\"ively expected to
grow exponentially with the number of elements, and there is substantial
diversity in essentially all individual element ratios.
This concept is fundamental to many observational projects underway, although
as reality is setting in, many of the projects and investigators are thinking
about probabilistic generalizations of this originally simple idea.

Going against this hope is the observation that there is probably not an infinite number
of nucleosynthetic and element-abundance-modifying processes at work in setting
stellar surface abundances:
There are various kinds of supernovae, and various nucleosynthetic and
spallation processes that create the elements prior to a star's formation,
and there are interstellar
and circumstellar processing mechanisms that perform element separation
in preparation for accretion at late times.
That is, even if there are 30 measured elements in a stellar photosphere,
there might only be $K$
important and relatively deterministic processes creating those 30 abundances.
In this case, the 30 element abundances for each star are only constraining that
star's location in a $K$-dimensional subspace of the 30-space.

Types of supernovae are few in number.

Types of nucleosynthetic process are few in number.

Open clusters show great homogeneity, so intra-cloud mixing must be good.

Therefore, stars cannot span a huge space in chemical abundances.

Indeed, this conclusion has been arrived at previously in theory, and
the dimensionality has been measured before in data. The measurements to
date suffer from significant issues: precision, methodology.

Here we test this with the best possible data set at the present day:
All the reasons these twins are good.

\section{Generalities}

This \documentname\ wants to ask the question:
\emph{What is the observed dimensionality of chemical-abundance
  space?}
Unfortunately---and like all important questions---this question is
ill posed.
Our problem in this \sectionname\ is to write down well-posed questions
that go some way towards answering the ill-posed question.

\section{Data}

Bedell!

Show (some of) the 900 plots here; they justify the precision claims.
I would say all 32 against feh and then all 32 against something that
clearly demonstrates the precision.

Ness cluster sample.

All of APOGEE RC sample?

\section{Experiments and results}

PCA first; it is highly suggestive. Take then the opportunity to cricize PCA.

Is there any sense in which the eigen-directions look like SNe yields?
Or nucleosynthetic yields from processes?

\section{Discussion}

\appendix
\section{Linear vs log space}

Almost all work on element abundances works in log (base 10), and most is
relative to Solar.
If we assume that there are a finite number $K$ of nucleosynthetic pathways,
and that each of those pathways produces elements in nearly fixed
ratios of atoms or mass, then the abundances will lie in a $K$-dimensional
subspace (or lower).
This subspace will only appear as a strictly linear subspace (that is, strictly
spanned by $K$ vectors) in linear ratios of elements.
These linear ratios can be mass ratios or atomic number ratios, and they can be ratioed
to other elements or Solar abundances, but without taking logarithms.

If the dimensionality of the element abundance linear subspace is $K$, and the
subspace is searched for with a linear algorithm but in the logarithmic element
abundances, then in principle more than $K$ significant dimensions or vectors
will be required to span the space of the data.
This is because a linear subspace in the linear data will obtain some curvature
when the logs are taken, and that curvature will be in directions that are orthogonal
to the subspace (in general).

For these reasons, in this paper, when we perform the dimensionality reduction,
we take the (log) abundances back to linear, perform the dimensionality reduction
in the linear space, and then transform the output so that it can be interpreted
logarithmically.
This latter transformation---of eigenvectors in the linear space to equivalents
in the log space---works as follows:

[Give equation and derivaton here.]

The amusing thing about all this, in the context of the project reported in this paper,
is that the spread in chemical abundances is very small in the sample.
In the limit of infinitesimal element-abundance spreads, there will be no differences
between the properly transformed linear eigenvectors and logarithmic space eigenvectors.
We demonstrate this in Figure~XXX.

\end{document}
